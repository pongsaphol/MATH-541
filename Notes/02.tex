\chapter{Subgroups}

\section{Definition and Examples}

\begin{definition}
  Let $G$ be a group. The subset $H$ of $G$ is a subgroup of $G$ if 
  \begin{itemize}
    \item $1_G \in H$
    \item $\forall x, y \in H$, $x \cdot y \in H$
    \item $\forall x \in H$, $x^{-1} \in H$
  \end{itemize}
  We write $H \leq G$ to indicate that $H$ is a subgroup of $G$.
\end{definition}

\begin{proposition}
  A subset $H$ of a group $G$ is a subgroup of $G$ if and only if
  \begin{itemize}
    \item $H \neq \emptyset$
    \item $\forall x, y \in H, xy^{-1} \in H$
  \end{itemize}
\end{proposition}

\section{Centralizers and Normalizers, Stabilizers and Kernels}

\begin{definition}[Centralizer]
  Let $G$ be a group and $A$ be a subset of $G$.
  The centralizer of $A$ in $G$ is
  $$C_G(A) = \{g\in G \mid gag^{-1} = a \text{ for all } a \in A\}$$
  Moreover, $C_G(A)$ is a subgroup of $G$.
\end{definition}

\begin{definition}[Center]
  Let $G$ be a group.
  The center of $G$ is 
  $$Z(G) = \{g \in G \mid gx=xg \text{ for all } x \in G\}$$
\end{definition}

\begin{definition}[Normalizer]
  Let $G$ be a group and $A$ be a subset of $G$.
  Let
  $$gAg^{-1} = \{gag^{-1}\mid a\in A\}$$
  The Normalizer of $A$ in $G$ is $$N_G(A) = \{g \in G \mid gAg^{-1} = A\}$$ 
\end{definition}

\begin{definition}[Stabilizer]
  If $G$ is a group acting on a set $S$ and $s$ is some fixed element of $S$ 
  the stabilizer of $s$ is 
  $$G_s = \{g \in G \mid g\cdot s = s\}$$
\end{definition}

\section{Cyclic groups}

\begin{definition}
  A group $H$ is a cyclic if $H$ can be generated by a single element. i.e., $H = \langle x \rangle = \{x^n \mid n \in \ZZ\}$ for some $x \in H$.
\end{definition}

\begin{proposition}
  If $H = \langle x \rangle$, then $|x| = n$.
\end{proposition}

\begin{proposition}
  Let $G$ be an arbitrary group, $x \in G$ and let $m, n \in \ZZ$. If $x^n = 1$ and
$x^m = 1$, then $x^d = 1$, where $d = (m, n)$.  
\end{proposition}

\begin{proof}
  By the Euclidean Algorithm, there exists $q, r \in \ZZ$ such that $d = mr + ns$ where $d = (m, n)$. Thus
  $$x^d = x^{mr+ns} = (x^{m})^r(x^n)^s = 1^r1^s=1$$
\end{proof}

\begin{theorem}
  If $H_1, H_2$ is cyclic groups and $|H_1| = |H_2|$ then $H_1 \cong H_2$.
\end{theorem}

\begin{proposition}
  Let $G$ be a group, let $x \in G$ and let $a \in \ZZ - \{0\}$.
  \begin{enumerate}
    \item If $|x| = \infty$, then $|x^a| = \infty$
    \item If $|x| = n < \infty$, then $|x^a| = \frac{n}{(n, a)}$
    \item 
  \end{enumerate} 
\end{proposition}

\begin{theorem}
  If $H = \langle x \rangle$ and $|x| = n$ then $x^a = 1$ if and only if $n \mid a$.
\end{theorem}

\begin{theorem}
  If $H = \langle x \rangle$ and $K \leq H$. Then $K$ is cyclic 
\end{theorem}

\begin{proof}
  Let $a$ be the least positive integer such that $x^a \in K$, let $y = x^a$
  
  Then we want to show $\langle y \rangle = K$. 
  \begin{itemize}
    \item $\langle y \rangle \subseteq K$ Obvious
    \item $\langle y \rangle \supseteq K$ Given $x^b \in K$ we can write $b = am + r$ with $a \leq r < a$ 
    \begin{align*}
      x^b &= x^{am+r} = (x^a)^mx^r \\
    &= y^m \underbrace{x^r}_{\in K} \\
    x^r &= \underbrace{y^{-m}}_{\in K}\underbrace{x^b}_{\in K}
    \end{align*}
    So, $x^r \in K$ so $r = 0$, $x^b = y^m$
  \end{itemize}
  Therefore $\langle y \rangle = K$
\end{proof}
