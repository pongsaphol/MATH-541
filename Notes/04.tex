\chapter{Group Actions}

% A group action $G$ on $A$ means

For any $g \in G$ we have 
\begin{align*}
  \sigma_g: A &\to A \\
  a &\mapsto g\cdot a 
\end{align*}

$\varphi: g \mapsto \sigma_g$

\begin{itemize}
  \item $g \cdot (h \cdot a) = gh \cdot a$
  \item $1 \cdot a = a$
\end{itemize}
\begin{example}
  $\varphi: G \to S_A$ is a homomorphism
  $g\cdot a$ = $\varphi(g)(a)$, then 
  $\varphi$ is homomorphism then $G$ action on A by $g \cdot a = \varphi(g)(a)$ 
\end{example}

\begin{proof}
  \begin{align*}
    g\cdot(h\cdot a) &= \varphi(g)(\varphi(h)(a)) \\
    &=(\varphi(g)\circ\varphi(h))(a) \\
    &=\varphi(gh)(a) \\
    &=gh\cdot a
  \end{align*}
  \begin{align*}
    1\cdot a &= \varphi(1)(a) \\
    &= 1_{S_A}(a) \\
    &=a
  \end{align*}
\end{proof}

\begin{example}
  The Kernel of the action $G$ on $A$ is 
  \begin{align*}
    \{g \in G \mid ga = a\ \forall a \in A\} &= \{g \mid \sigma_g = id_A = 1_{S_A}\} \\
  \end{align*}
\end{example}

\begin{example}
  For each $a \in A$, the stabilizer of $a$ is
  \[G_a = \{g \in G \mid g \cdot a = a\}\]
\end{example}

\begin{observation}
  If $G$ acts on $A$, faithfully then
  \[G \cong \varphi(G)\]
\end{observation}

\begin{proof}
  $G \cong G/\ker(\varphi) \cong \varphi(G) \subseteq S_A$
\end{proof}


\section{Group Actions and Permutation Representations}

\begin{definition}
  Let $G$ be a group, $\varphi: G \to S_A$ is a ``permutation representation'' of $G$ into $S_A$
\end{definition}

\begin{proposition}[Orbit Equivalence Relations]
  Let $G$ acts on $A$. Define the relation $\sim$ on $A$ by
  $a \sim b$ if $a = gb$ for some $g\in G$
  Then $\sim$ is an equivalence relation

  For each $a \in A$, $|[a]|$ is $|G : G_a|$
\end{proposition}

\begin{proof}
  Check $\sim$
  \begin{itemize}
    \item (reflexive $a \sim a$) $1_G\cdot a = a$
    \item (symmetric $a \sim b \implies b \sim a$) $\sigma_(g^{-1}) = (\sigma_g)^{-1}$
    \item (transitive $a \sim b \land b \sim c \implies a \sim c$) 
    \begin{align*}
      h\cdot b &= a \\
      h\cdot g\cdot c &= a \\
      (hg)\cdot c &= a \\
    \end{align*}
  \end{itemize}

  Every element of $G$  
\end{proof}

\begin{definition}
  Let $G$ acts on $A$. 
  \begin{itemize}
    \item $[a] = \{b \mid a \sim b\}$ is called the orbit of $G$ containing $a$
    \item $a \sim b$ is said ``$a$ and $b$ are equivalent''
    \item The action of $G$ on $A$ is \textbf{transitive} if there is only 1 orbit (orbit class)
  \end{itemize}
\end{definition}

\subsection{cycle decomposition}

\begin{example}
  Every element $\sigma \in S_A$ has a cycle decomposition
  \[\sigma = (a_1\ a_2\ \dotsc)(b_1\ b_2\ \dotsc)\dotsc\]
\end{example}

\begin{theorem}
  Cycle decomposition are unique up to  permutiting between cycles and rotating the cycles
\end{theorem}

\begin{theorem}
  If $\sigma \in S_A$, then $\sigma$ is a product of distinct elements $(n\ x)$ or $(n\ y)$ for $n \in A, x,y \notin A$
\end{theorem}

\section{The left-multiplication action()}
$G$ acts on $G$ by $g\cdot h = gh$, $g_1(g_2\cdot h) = g_1g_2h = g_1g_2\cdot h$

\begin{observation}
  The left-multiplication action is transitive, faithful and $G_a = \{1\}$ for any $a \in G$
\end{observation}

\begin{proof}
  $(ba^{-1})\cdot a = b$, so $ba^{-1}$ moves $a$ to $b$ so, the action is transitive.
  $$x \in G_a \iff x \cdot a = a \iff x a a^{-1} = a^{-1} \iff x = 1$$
\end{proof}

\begin{theorem}
  $H \le G$, $G$ act on $A$ by left-multiplication
  \begin{enumerate}
    \item $G$ acts transitively on $A$
    \item $G_{1H} = H$
    \item $\ker$ of the action ($= \{g \in G \mid g \cdot aH = aH\}$) 
    = $\bigcap_{x \in G}xHx^{-1} = $the normal subgroup of $G$ which is contained in $H$
  \end{enumerate}
\end{theorem}

\begin{proof}
  \begin{enumerate}
    \item $(ba^{-1})\cdot aH = bH$
    \item \begin{align*}
      G_{1H} &= \{g \in G \mid g\cdot 1H = 1H\} \\ 
        &= \{g \in G \mid gH = 1H\} \\
        &= \{g \in 1^{-1}g \in H\} \\
        &= \{g \in g \in H\} \\
        &= H
    \end{align*}
    \item \begin{align*}
      g \in \ker(\text{action}) &\iff g\cdot x H = xH\ \text{ for every } x\in G \\
      &\iff g \in \bigcap_{x \in G} \{a \in G \mid a\cdot xH = xH\} \\
      &\iff g \in \bigcap_{x \in G} \{a \in G \mid x^{-1}a x \in H\} \\
      &\iff g \in \bigcap_{x \in G} \{a \in G \mid a \in xH x^{-1}\} \\
      &\iff g \in \bigcap_{x \in G} xHx^{-1}
    \end{align*}
  \end{enumerate}
\end{proof}

\begin{corollary}
  If $G$ is a finite group of order $n$. Let $p$ be the smallest prime dividing $n$.
  Suppose $H \le G$ so $|G:H| = p$ then $H \trianglelefteq G$
\end{corollary}
\begin{proof}
  $G \curvearrowright A = \{\text{left }H\text{-cosets}\}$ 
  
  $\pi: G \to S_A$ be the representation of this action

  $k = \ker(\pi)$

  Goal: $H = K$, i.e., $|H:K|=1$
\end{proof}

\section{Conjugation action of $G$ on $G$}

\[g\cdot h = ghg^{-1}\]
\[g\cdot S = gSg^{-1} = \{gxg^{-1} \mid x \in S\}\]

Size of an orbit, orbit of a $=|G:G_a|$

Size of orbit of $h \in G = |G:C_G(h)|$, $S \subseteq G = |G:N_G(S)|$

\begin{theorem}[the class equation]
  let $G$ be a finite group and $g_1, g_2, \dotsc g_r$ be reoresentatives of all the conjugacy classes
  then \[|G| = \sum_{i = 1}^r |\text{Orbit}(g_i)| = \sum_{i=1}^r|G:G_{g_i}| = \sum_{i=1}^r|G:C_G(g_i)|\]
\end{theorem}

let $g_1, \dotsc, g_r$ be representative of every conjugate class not contained in $Z(G)$
\[|G| = \sum_{i=1}^r|G:C_G(g_i)| + |Z(G)|\]

\begin{theorem}
  let $p$ be prime, $G$ a group of order $p^\alpha$ for some $\alpha \in \NN$ then 
  $|Z(G)| \neq 1$
\end{theorem}
\begin{proof}
  let $g_1, \dotsc, g_r$ represents all conjugacy classes of size $\ge 1$
  \begin{align*}
    p^\alpha &= |G| = \sum_{i=1}^r |g:C_G(g_i)| + |Z(G)| \\
    |G| &= \underbrace{|G:C_G(g_i)|}_{\neq 1} \cdot |C_G(g_i)|
  \end{align*}
  So, $p \mid |G:C_G(g_i)|$, so $p \mid |Z(G)| \implies |Z(G)| \ge p$
\end{proof}


\begin{example}
  
Q:What is a conjugacy class in $S_n$?

Describe when 2 diff, products of disj cycles are conjugate

$\sigma = (1\ 2\ 3)(4\ 5)(6)$, for $\tau = (1\ 2\ 3\ 4\ 5\ 6)$
\begin{align*}
  \tau\sigma\tau^{-1} &= (1)(2\ 3\ 4)(5\ 6) \\
  \sigma &= (6)(1\ 2\ 3)(4\ 5) \\
  \tau\sigma\tau^{-1} &= (\tau(6))(\tau(1)\ \tau(2)\ \tau(3))(\tau(4)\ \tau(5)) 
\end{align*}
\end{example}

\begin{observation}
  If $\sigma = (a_1\ \dotsc a_{r_1})(a_{r_1+1}\ \dotsc\ a_{r_2})\dotsc$
  then 
  \[\tau\sigma\tau^{-1} = (\tau(a_1)\ \dotsc \tau(a_{r_1}) \dotsc\]
\end{observation}
\begin{proof}
  \begin{align*}
    \tau\sigma\tau^{-1}(\tau(a_j)) &= \tau\sigma(a_j) = \tau(a_{j_1}) \\
    \tau\sigma\tau^{-1}(\tau(a_{r_n})) &= \tau\sigma(a_{r_n}) = \tau(a_{r_{n-1}} + 1)
  \end{align*}
\end{proof}

If $\sigma = $ a product of disj cycles of lengths $n_1, \dotsc, n_r$ (including 1-cycles)
and $n_1 \le \dotsc \le n_r$ then $(n_1\ \dotsc\ n_r)$ is the cycle-type of $\sigma$

\begin{observation}
  $p$ is conj to $\sigma$ $\iff$ they have the same cycle-type.
\end{observation}

\begin{example}
  The conj classes in $S_5$
\end{example}