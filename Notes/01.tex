
\begin{itemize}
  \item Book: Dujmit Foote ``Modern Algebra 3rd ed''
  \item Midterm 3/23 in class
  \item Final 5/8
  \item Homeworks: ~weekly
  \item Honors Credit: Extra sections + homeworks
\end{itemize}


\chapter{Groups}
Operations often modeled:
$+, \cdot$

composition: space of thing that you are looking at $\leftarrow$ alomst always not commutative

\textbf{Groups}: One operation $\cdot$

\textbf{Rings}: 2 operations: $+, \cdot$ that play nice


\section{Axioms of Groups}

By ``operation'' on $S$, I mean a function $\cdot S \times S \lthen S$

Instead of $\cdot(a, b)$, we write $a \cdot b$

A group is a set $G$ with an operation $\cdot$ satisfying:
\begin{enumerate}
  \item Associativity: $(a \cdot b) \cdot c = a \cdot (b \cdot c)$
  \item There is an identity element: there is one special element $1 \in G$ so $1 \cdot a  = a$ for any $a \in G$
  and $a \cdot 1 = a$ for any $a \in G$
  \item Inverses: For any $a \in G$, there is a $b \in G$ so $a \cdot b = b \cdot a = 1$

\end{enumerate}

\textbf{Note}: $a \cdot b = b\cdot a$ is \underline{not} an axiom.

If $G$ satisfies this, we call it an abelian group

\begin{example}
  $(\ZZ, +), (\QQ, +), (\RR, +), (\CC, +)$
\end{example}

\begin{enumerate}
  \item $0$ is the identity
  \item inverses: $-a$ is the inverse of $a$
\end{enumerate}

\begin{example}
  $(\QQ \setminus \{0\}, \cdot), (\RR \setminus \{0\}, \cdot), (\CC \setminus \{0\}, \cdot)$ 
\end{example}

\begin{enumerate}
  \item $1$ is the identity
  \item Inverses: $\frac{1}{a}$ is the inverse of $a$
\end{enumerate}

\textbf{Note:} $(\ZZ \setminus \{0\}, \cdot)$ is not a group

$(V, +)$ is a group

\begin{example}
  For $n$, a natural number, $(\ZZ/n\ZZ, +)$ is a group
\end{example}

On $\ZZ$, we say $a, b$ are $(\text{mod } n)$ equivalent (written $a \equiv b (\text{mod } n)$) 
if $n$ divides $a-b$

$\ZZ/n\ZZ$ is the set of equivalence classes mod $n$

\begin{example}
  $n = 2$: (odds, evens) which is $\{0_{\text{mod } 2}, 1_{\text{mod } 2}\}$
\end{example}

$17_{\text{mod } 2} + 64_{\text{mod } 2} = 81_{\text{mod } 2} = 1_{\text{mod } 2}$

\begin{example}
  $\ZZ/3\ZZ = \{0_{\text{mod } 3}, 1_{\text{mod } 3}, 2_{\text{mod } 3}\}$
\end{example}

\begin{example}
  $(2\ZZ, +)$ is a group (even numbers)
\end{example}

\begin{example}
  If $(G, \cdot_G)$ and $(H, \cdot_H)$ are groups, then $(G \times H, \cdot_{G} \times \cdot_H)$ is a group
\end{example}

\begin{itemize}
  \item $(g_1, h_1) \cdot_{G \times H} (g_2, h_2) = (g_1 \cdot_G g_2, h_1 \cdot_H h_2)$
  \item Identity: $1_{G \times H} = (1_G, 1_H)$
  \item Inverse of $(g, h)$: $(g^{-1}, h^{-1})$ 
\end{itemize}

\subsection{Properties}
\begin{itemize}
  \item $G$ has exactly 1 identity
  \item Each $g \in G$, there is exactly 1 inverse of $g$ we write this $g^{-1}$
  (i.e. $^{-1}: G \rightarrow G$)
  \item $(g^{-1})^{-1} = g$
  \item $(a \cdot b)^{-1} = b^{-1} \cdot a^{-1}$
  \item $(a_1 \cdot a_2 \cdot \dotsc \cdot a_m)^{-1} = a_m^{-1} \cdot a_{m-1}^{-1} \cdot \dotsc \cdot a_1^{-1}$
\end{itemize}

\begin{proof}
  \text{}
  \begin{itemize}
    \item Suppose $a, b$ are both identities in $G$. Then $a = a \cdot b = b$
    \item Suppose $a, b$ are both inverses of $g$. i.e $a\cdot g = g\cdot a = 1$ and 
    $b\cdot g = g\cdot b = 1$ Then $b = 1\cdot b = (a\cdot g)\cdot b = a\cdot(g\cdot b) = a\cdot 1 = a$ 
    \item know $g\cdot g^{-1} = g^{-1}\cdot g = 1$ so $(g^{-1})^{-1} = g$
    \item $(a \cdot b)^{-1}$ satisfies: $x\cdot(a\cdot b) = (a\cdot b)\cdot x = 1$ 
    we check $b^{-1}a^{-1}$ does this 

    $(b^{-1}a^{-1})\cdot(a\cdot b) = b^{-1}(a^{-1}\cdot a)b = b^{-1} \cdot 1 \cdot b = b^{-1}b = 1$

    $(ab)(b^{-1}a^{-1}) = a(b\cdot(b^{-1})\cdot a^{-1} = a \cdot 1 \cdot a^{-1} = aa^{-1} = 1$
  \end{itemize}
\end{proof}

\begin{theorem}
  In $G$, there is exactly 1 solution to the equation $ax = b$ for a fixed $a, b \in G$
\end{theorem}

\begin{corollary}
  Cancellation laws:
  \[ax = ay \implies x = y\]
  \[xa = ya \implies x = y\]
\end{corollary}

\begin{proof}
  If $a\cdot x = b$
  \begin{align*}
    a^{-1}\cdot a\cdot x &= a^{-1}\cdot b \\
    (a^{-1}\cdot a)\cdot x &= a^{-1}\cdot b \\
    1x = x &= a^{-1}\cdot b
  \end{align*}
\end{proof}

\begin{definition}
  For $x \in G$, the order of $x$, written $|x|$, is the least $n > 0$ so

  \[x^n = \underbrace{x\cdot x\cdot \dotsc \cdot x}_n = 1_G\]
\end{definition}
If there is no such $n$, $x$ has ``infinite order''
\begin{example}
  In $(\RR \setminus \{0\}, \cdot)$, $|5| = \infty$, $|-1| = 2$, $|1| = 1$
\end{example}

\begin{example}
  $(\ZZ/6\ZZ, +)$, $|1_{\text{mod } 6}| = 6$, $|2_{\text{mod } 6}| = 3$, $|3_{\text{mod } 6}| = 2$, $|4_{\text{mod } 6}| = 3$, $|5_{\text{mod } 6}| = 2$
\end{example}

\section{Dihedral Groups}

\subsection{Triangle}

Look at the collection of symmetries of an equilateral Triangle

\begin{center}
\begin{tikzpicture}[scale=2]
  
  \draw (0,0) node[anchor=north]{$1$}
  -- (2,0) node[anchor=north]{$3$}
  -- (1,1.732) node[anchor=south]{$2$}
  -- cycle;
\end{tikzpicture}
\end{center}

\begin{multicols}{3}
Rotation right
\begin{itemize}
  \item 1 $\rightarrow$ 2
  \item 2 $\rightarrow$ 3
  \item 3 $\rightarrow$ 1
\end{itemize}
$r$
\par

Rotation Left
\begin{itemize}
  \item 1 $\rightarrow$ 3
  \item 2 $\rightarrow$ 1
  \item 3 $\rightarrow$ 2
\end{itemize}
$r^2$

\par
Reflection around 2
\begin{itemize}
  \item 1 $\rightarrow$ 3
  \item 2 $\rightarrow$ 2
  \item 3 $\rightarrow$ 1
\end{itemize}
$r^2 \circ s$

\end{multicols}

\begin{multicols}{3}
Reflection around 1
\begin{itemize}
  \item 1 $\rightarrow$ 1
  \item 2 $\rightarrow$ 3
  \item 3 $\rightarrow$ 2
\end{itemize}
$s$

\par
Reflection around 3
\begin{itemize}
  \item 1 $\rightarrow$ 2
  \item 2 $\rightarrow$ 1
  \item 3 $\rightarrow$ 3
\end{itemize}
$r \circ s = s \circ r^2$
\par
Identity 
\begin{itemize}
  \item 1 $\rightarrow$ 1
  \item 2 $\rightarrow$ 2
  \item 3 $\rightarrow$ 3
\end{itemize}
$r^3, s^2$
\end{multicols}
\begin{align*}
  r^2s &= r\cdot(r\cdot s) \\ 
  &= (r\cdot s) \cdot r^{-1} \\
  &= s \cdot (r^{-1} \cdot r^{-1}) \\
  &= s \cdot (r^{-1})^2
\end{align*}


(Symmetry of $\triangle, \circ$) = $D_6$
\pagebreak
\subsection{n-gon}


\begin{multicols}{3}
Rotation right
\begin{itemize}
  \item $k \rightarrow k + 1$ (for $k < n$)
  \item $n \rightarrow 1$
\end{itemize}
$r, |r| = n$
\par
Reflection around 1
\begin{itemize}
  \item $k \rightarrow n + 2 - k$
  \item $1 \rightarrow 1$
\end{itemize}
$s, |r| = n$
\par
My Symmetry
\begin{itemize}
  \item $1 \rightarrow k$
  \item $2 \rightarrow k + 1$
\end{itemize}
$r^k$
\end{multicols}
So, $\{r, s\}$ generates the group of sym of regular n-gon

(Symmetry of a regular n-gon, $\circ$) = $D_{2n}$

\subsection{Definition}

Rules of dihedral group multiplication in $D_{2n}$
$\{r, s\}$

\begin{enumerate}[a)]
  \item $r^n = 1$
  \item $s^2 = 1$
  \item $r \cdot s = s \cdot r^{-1}$
\end{enumerate}

When you have generators $S$ for $G$ and can list $R_1, R_2, R_3$ 
\underline{all} the rules you need to know to do multiplication in $G$
Then $\langle S, R_1, R_2, R_3\rangle$ is a ``presentation of the group $G$''

$D_{2n} = \langle r, s \mid r^n = 1, s^2 = 1, rs = sr^{-1}\rangle = \{1, r, \dotsc, r^{n-1}, s, rs, \dotsc, rs^{n-1}\}$

Fact: There is a finite set of rule $R_1, \dotsc, R_{2000}$
so $\langle a, b|R_1, \dotsc, R_{2000}\rangle$ ``undecidable word problem''

\section{Symmetric Group}

Given $\Omega$ any set, $S_\Omega = $ The permutations of $\Omega =$ The bijections $f: \Omega \to \Omega$

\begin{example}
  $\Omega = \{1, 2, 3\}$ 

  $S_n = S_{\{1, 2, \dotsc, n\}}$ has $n!$ elements

  $|S_3| = 6, |D_6| = 6, D_6 \subseteq S_3$

  $|D_{2n}| = 2n$

  $|S_n| = n!$
\end{example}

\subsection{Cycle Decomposition}

$1 \to 4, 2 \to 1, 3 \to 2, 4 \to 3, 5 \to 5$ can be written as
$(1432)(5)$

$(a_1 \dotsc a_{m_1})(a_{m_1+1} \dotsc a_{m_2})$ with $a_i$ is disjoint
represents the function which satisfies

\begin{itemize}
  \item $a_i$ to $a_{i+1}$ unless $i = m_j$ for some $j$
  \item $a_{m_j}$ to $a_{m_{j-1}}+1$ $j \neq 1$
  \item $a_{m_1}$ to $a_1$
\end{itemize}

$(1)(2)(3)(4)(5)(6)(7) = 1$

$(1442)\circ(3421)=(124)$

$|(123)(45)|=6$

Order of a product of disjoint cycles is the lcm(lengths of the cycles)

\section{Homomorphisms and Isomorphisms}

\begin{definition}
  A homorphism from $(G, \cdot_G)$ to $(H, \cdot_H)$ is a function
  $f: G \to H$ such that $$f(x \cdot_G y) = f(x) \cdot_H f(y)$$ for all $x, y \in G$  
\end{definition}

\begin{itemize}
  \item $f(x^{-1}) = f(x)^{-1}$
  \begin{align*}
    f(x) &= f(1_G \cdot_G x) \\
    &= f(1_G) \cdot_H f(x) \\
    f(x) \cdot_H (f(x))^{-1} &= f(1_G) \cdot_H f(x) \cdot_H (f(x))^{-1} \\
    1_H &= f(1_G)
  \end{align*}
  \[1_H = f(1_G) = f(x \cdot_G x^{-1}) = f(x)\cdot_H f(x^{-1})\]
  \[1_H = f(1_G) = f(x^{-1} \cdot_G x) = f(x^{-1})\cdot_H f(x)\]
\end{itemize}

\begin{definition}
  If $f$ is a bijection and a homorphism, then $f$ is an isomorphism
\end{definition}

\begin{example}
    $\cdot id: G \to G$

    $\cdot^{-1}: G \to G, x \mapsto x^{-1}$

    $(x \cdot y)^{-1} = (x^{-1}) \cdot (y^{-1})$

    is an isomorphism if and only if $G$ is abelian

    $xyx^{-1}y^{-1} = 1$
\end{example}

\begin{example}
  $e^x: (\RR, +) \to (\RR, \cdot), f(x+y)=f(x)\cdot f(y)$
  is an isomorphism
\end{example}

\begin{example}
  $f : \ZZ/6\ZZ \to \ZZ/3\ZZ$
  \begin{itemize}
    \item $0 \to 0$
    \item $1 \to 1$
    \item $2 \to 2$
    \item $3 \to 0$
    \item $4 \to 1$
    \item $5 \to 2$
  \end{itemize}
  is a homorphism NOT an isomorphism
\end{example}

\begin{definition}
  $G$ and $H$ is isomorphic if there is a $f: G \to H$ which is an isomorphism 
  (written $G\cong H$)
\end{definition}

If $G\cong H$ then 
\begin{itemize}
  \item $G$ is a belian iff $H$ is abelian
\end{itemize}

\textbf{Abelian:} For every $x, y$ $x\cdot_G y = y\cdot_G x$
\[f(y)\cdot_H f(x) = f(y\cdot_G x) = f(x\cdot_G y) = f(x)\cdot_Hf(y)\]
So, any 2 elements

If $f$ is a $\cong$, $f: G \to H$ and $x \in G$ has order 2
Then $f(x) \in H$ has order 2
\begin{align*}
  x^2 &= 1_G \\
  (f(x))^2 = f(x)\cdot f(x) &= f(x\cdot x) = f(1_G) = 1_H
\end{align*}

\textbf{Recall} $D_{2n} = \langle r, s \mid r^n = 1, s^2 = 1, rs=sr^{-1}\rangle$

If $G = \langle g_1, \dotsc, g_n \mid R_1, R_2, \dotsc \rangle$ and 
$h_1, \dotsc, h_n \in H$ so $R_1(h_1\dotsc h_n)\dotsc$ 
Then $f: g_i \mapsto h_i$ is a homomorphism 

\section{Group Actions}
\begin{definition}
  A group action is a function
  \[\alpha: G \times A \to A\]
  so \[\alpha(g, \alpha(h, a)) = \alpha(g\cdot h, a)\]
  We write $g \cdot a$ for $\alpha(g, a)$
  \[g\cdot(h\cdot a) = (g\cdot h)\cdot a\]
\end{definition}

\begin{itemize}
  \item $1_G \cdot a = a$ for any $a \in A$
\end{itemize}

For any $g \in G$ the function $g\cdot: A \to A$, $a \mapsto g\cdot a$ is a bijection of $a$.
\begin{align*}
(g\cdot(g^{-1}\cdot_G))&: A \to A\\
  &= (g\cdot g^{-1})\cdot a\\
  &= 1_G\cdot a = a\\
  g^{-1}(g\cdot a) &= a
\end{align*} 

Since this function has an inverse (as a funciton) it is bijective

\textbf{Recall: } $S_A$ is the group of all permutations of $A$

Get a function $\sigma: G \to S_A$ and $\sigma(g) = $ the function $a \mapsto g\cdot a$

\textbf{Observation: }$\sigma$ is a homomorphism
\[\sigma(g\cdot h) = \sigma(g)\cdot\sigma(h)\]

\begin{example}
  $(\RR, +)$ acts on $A = \{1, 2, 3\}$ 

  $g\cdot a = a$

  $\sigma: \RR \to S_3$, $g \mapsto 1_{S_3}$
\end{example}