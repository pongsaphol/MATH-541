\chapter{Quotient Groups and Homomorphisms}

\section{Definition and Examples}

\begin{definition}
  If $\varphi: G \to H$ is a homomorphism then $\ker(\varphi) = \{x \in G \mid \varphi(x) = 1_H\}$
\end{definition}

\begin{lemma}
  $\ker(\varphi) \le G$
\end{lemma}

\begin{proof}
  Proof eash properties of subgroup 
  \begin{itemize}
    \item Closed identity, Since $\varphi(1_G) = 1_H$ 
    \begin{align*}
      \varphi(1_G) = \varphi(1_G1_G) = \varphi(1_G)\cdot\varphi(1_G) = 1
    \end{align*}
    So, $1_G \in \ker(\varphi)$
    \item Closed under inverses, if $x \in ker(\varphi)$
    \begin{align*}
      \varphi(x^{-1}) &= (\varphi(x))^{-1} = (1_H)^{-1} = 1_H \\
      1_H = \varphi(1_G) &= \varphi(x^{-1}x) = \varphi(x)\cdot \varphi(x^{-1})\\
    \end{align*}
    So, $x^{-1} \in \ker(\varphi)$
    \item Closed under multiplication, if $x,y \in \ker(\varphi)$
    \begin{align*}
      \varphi(xy) &= \varphi(x) \cdot \varphi(y) \\
      &= 1_H \cdot 1_H = 1_H
    \end{align*}
    So, $xy \in \ker(\varphi)$
  \end{itemize}
\end{proof}

\begin{definition}
  Given $\varphi: G \to H$ a homomorphism and $K = \ker(\varphi)$ 
  For any $a \in H$, let $$X_a = \{x \in G \mid \varphi(x) = a\}$$
  then
  $$G/K = (\{X_a \mid a \in H\}, \circ)$$ where $$X_a \circ X_b = X_{ab}$$
\end{definition}

\begin{lemma}
  If $\varphi: G \to H$ is a homomorphism, $K = \ker(\varphi)$, and $\varphi(b) = a$ 
  then $X_a = bK$ where $bK = \{b\cdot z \mid z \in K\}$
\end{lemma}

\begin{proof}
  The goal is to show $X_a = bK$
  \begin{itemize}
    \item $X_a \supseteq bK$, Given $y \in bK, y = b\cdot z$ for some $z \in K$
    \[\varphi(y) = \varphi(b\cdot z) = \varphi(b)\cdot\varphi(z) = a \cdot 1_H = a\]
    \item $X_a \subseteq bK$, Given $\varphi(y) = a$
    \[\varphi(b^{-1}y) = \varphi(b^{-1})\varphi(y) = (\varphi(b))^{-1}\cdot\varphi(y) = a^{-1}\cdot a = 1\]
  \end{itemize}
  Therefore $X_a = bK$
\end{proof}

% \begin{proposition}
%   Let $\varphi: G \to H$ be a homomorphism of groups with kernel $K$. Let $X_a \in G/K$ be the fiber above $a$, 
%   i.e., $X_a = \varphi^{-1}(a)$ Then
%   \begin{enumerate}
%     \item For any $u \in X_a, X_a = \{uk \mid k \in K\}$
%     \item For any $u \in X_a, X_a = \{ku \mid k \in K\}$
%   \end{enumerate}
% \end{proposition}

% \begin{proof}

% \end{proof}

\begin{definition}
  For any $N \le G$ and for any $g \in G$ let 
  \[gN = \{gn \mid n \in N\}\]
  and 
  \[Ng = \{ng \mid n \in N\}\]
\end{definition}

\begin{theorem}
  Let $G$ be a group and $K$ be the kernel of some homomorphism. 
  Then the set whose elements are the left cosets of $K$ in $G$ with operation defined by
  \[uK \circ vK = (uv)K\]
  forms a group $G/K$.
\end{theorem}

\begin{proof}
  Let $X, Y \in G/K$ and let $Z = XY$ in $G/K$. Since $K$ is the kernel of some homomorphism,
  $\varphi: G \to H$, so $X = \varphi^{-1}(a)$ and $Y = \varphi^{-1}(b)$ for some $a, b \in H$.
  By definition of the operation in $G/K$, $Z = \varphi^{-1}(ab)$. 

  Let $u, v$ be arbitrary representatives of $X, Y$ ($\varphi(u) = a, \varphi(v)=b$ and $X = uK, Y = vK$)

  GOAL: show $uv \in Z$
  \begin{align*}
    uv \in Z &\iff uv \in \varphi^{-1}(a, b) \\
    &\iff \varphi(uv) = ab \\
    &\iff \varphi(u)\varphi(v) = ab \\ 
  \end{align*}
  Therefore $Z$ is the (left) coset $(uv)K$.
\end{proof}

\begin{proposition}
  If $N \le G$ then for all $u, v \in G, uN = vN$ if and only if $v^{-1}u\in N$
\end{proposition}

\begin{proof}
  Since $N$ is a subgroup of $G$, since $1_G \in N$ then 
  \[G = \bigcup_{g \in G} gN\]
  If $x \in uN \cap vN$ then for some $n_1, n_2 \in N$
  \begin{align*}
    x = un_1 &= vn_2 \\ 
    v^{-1}u &= n_2 n_1^{-1} \in N
  \end{align*}
  For any $n \in N$
  \[un = (vv^{-1})un = v(v^{-1}un) \in vN\]
  So $uN \subseteq vN$, wlog, $vN \subseteq uN$. Therefore $uN = vN$.
\end{proof}

\begin{definition}
  The element $gng^{-1}$ is called the \textit{conjugate} of $n \in N$ by $g$.
  The set $gNg^{-1}$ is called the \textit{conjugate} of $N$ by $g$.
  if $gNg^{-1} = N$ then $g$ is said to \textit{normalize} $N$. 

  If $N \le G$ called \textit{normal} if for any $g \in G$ normalizes $N$. In another word, $gNg^{-1} = N$ for all $g \in G$, written
  \[N \trianglelefteq G\]
\end{definition}

\begin{theorem}
  For $N \le G$, the following are equivalent
  \begin{itemize}
    \item $N \trianglelefteq G$
    \item $N_G(N) = G$
    \item $gN = Ng$ for all $g \in G$

    \item $gNg^{-1} \subseteq N$ for all $g \in G$
  \end{itemize}
\end{theorem}