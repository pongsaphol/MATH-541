\chapter{Quotient Groups and Homomorphisms}

\section{Definition and Examples}

\begin{definition}
  If $\varphi: G \to H$ is a homomorphism then $\ker(\varphi) = \{x \in G \mid \varphi(x) = 1_H\}$
\end{definition}

\begin{lemma}
  $\ker(\varphi) \le G$
\end{lemma}

\begin{proof}
  Proof eash properties of subgroup 
  \begin{itemize}
    \item Closed identity, Since $\varphi(1_G) = 1_H$ 
    \begin{align*}
      \varphi(1_G) = \varphi(1_G1_G) = \varphi(1_G)\cdot\varphi(1_G) = 1
    \end{align*}
    So, $1_G \in \ker(\varphi)$
    \item Closed under inverses, if $x \in ker(\varphi)$
    \begin{align*}
      \varphi(x^{-1}) &= (\varphi(x))^{-1} = (1_H)^{-1} = 1_H \\
      1_H = \varphi(1_G) &= \varphi(x^{-1}x) = \varphi(x)\cdot \varphi(x^{-1})\\
    \end{align*}
    So, $x^{-1} \in \ker(\varphi)$
    \item Closed under multiplication, if $x,y \in \ker(\varphi)$
    \begin{align*}
      \varphi(xy) &= \varphi(x) \cdot \varphi(y) \\
      &= 1_H \cdot 1_H = 1_H
    \end{align*}
    So, $xy \in \ker(\varphi)$
  \end{itemize}
\end{proof}

\begin{definition}
  Given $\varphi: G \to H$ a homomorphism and $K = \ker(\varphi)$ 
  For any $a \in H$, let $$X_a = \{x \in G \mid \varphi(x) = a\}$$
  then
  $$G/K = (\{X_a \mid a \in H\}, \circ)$$ where $$X_a \circ X_b = X_{ab}$$
\end{definition}

\begin{lemma}
  If $\varphi: G \to H$ is a homomorphism, $K = \ker(\varphi)$, and $\varphi(b) = a$ 
  then $X_a = bK$ where $bK = \{bz \mid z \in K\}$
\end{lemma}

\begin{proof}
  The goal is to show $X_a = bK$
  \begin{itemize}
    \item $X_a \supseteq bK$, Given $y \in bK, y = bz$ for some $z \in K$
    \[\varphi(y) = \varphi(b\cdot z) = \varphi(b)\cdot\varphi(z) = a \cdot 1_H = a\]
    \item $X_a \subseteq bK$, Given $\varphi(y) = a$
    \[\varphi(b^{-1}y) = \varphi(b^{-1})\varphi(y) = (\varphi(b))^{-1}\cdot\varphi(y) = a^{-1}\cdot a = 1\]
  \end{itemize}
  Therefore $X_a = bK$
\end{proof}

% \begin{proposition}
%   Let $\varphi: G \to H$ be a homomorphism of groups with kernel $K$. Let $X_a \in G/K$ be the fiber above $a$, 
%   i.e., $X_a = \varphi^{-1}(a)$ Then
%   \begin{enumerate}
%     \item For any $u \in X_a, X_a = \{uk \mid k \in K\}$
%     \item For any $u \in X_a, X_a = \{ku \mid k \in K\}$
%   \end{enumerate}
% \end{proposition}

% \begin{proof}

% \end{proof}

\begin{definition}
  For any $N \le G$ and for any $g \in G$ let 
  \[gN = \{gn \mid n \in N\}\]
  and 
  \[Ng = \{ng \mid n \in N\}\]
\end{definition}

\begin{theorem}
  Let $G$ be a group and $K$ be the kernel of some homomorphism. 
  Then the set whose elements are the left cosets of $K$ in $G$ with operation defined by
  \[uK \circ vK = (uv)K\]
  forms a group $G/K$.
\end{theorem}

\begin{proof}
  Let $X, Y \in G/K$ and let $Z = XY$ in $G/K$. Since $K$ is the kernel of some homomorphism,
  $\varphi: G \to H$, so $X = \varphi^{-1}(a)$ and $Y = \varphi^{-1}(b)$ for some $a, b \in H$.
  By definition of the operation in $G/K$, $Z = \varphi^{-1}(ab)$. 

  Let $u, v$ be arbitrary representatives of $X, Y$ ($\varphi(u) = a, \varphi(v)=b$ and $X = uK, Y = vK$)

  GOAL: show $uv \in Z$
  \begin{align*}
    uv \in Z &\iff uv \in \varphi^{-1}(a, b) \\
    &\iff \varphi(uv) = ab \\
    &\iff \varphi(u)\varphi(v) = ab \\ 
  \end{align*}
  Therefore $Z$ is the (left) coset $(uv)K$.
\end{proof}

\begin{proposition}
  If $N \le G$ then for all $u, v \in G, uN = vN$ if and only if $v^{-1}u\in N$
\end{proposition}

\begin{proof}
  Since $N$ is a subgroup of $G$, since $1_G \in N$ then 
  \[G = \bigcup_{g \in G} gN\]
  If $x \in uN \cap vN$ then for some $n_1, n_2 \in N$
  \begin{align*}
    x = un_1 &= vn_2 \\ 
    v^{-1}u &= n_2 n_1^{-1} \in N
  \end{align*}
  For any $n \in N$
  \[un = (vv^{-1})un = v(v^{-1}un) \in vN\]
  So $uN \subseteq vN$, wlog, $vN \subseteq uN$. Therefore $uN = vN$.
\end{proof}

\begin{definition}
  The element $gng^{-1}$ is called the \textit{conjugate} of $n \in N$ by $g$.
  The set $gNg^{-1}$ is called the \textit{conjugate} of $N$ by $g$.
  if $gNg^{-1} = N$ then $g$ is said to \textit{normalize} $N$. 

  If $N \le G$ called \textit{normal} if for any $g \in G$ normalizes $N$. In another word, $gNg^{-1} = N$ for all $g \in G$, written
  \[N \trianglelefteq G\]
\end{definition}

\begin{theorem}
  For $N \le G$, the following are equivalent
  \begin{itemize}
    \item $N \trianglelefteq G$
    \item $N_G(N) = G$
    \item $gN = Ng$ for all $g \in G$

    \item $gNg^{-1} \subseteq N$ for all $g \in G$
  \end{itemize}
\end{theorem}

\section{More on Cosets and Lagrange's Theorem}

\begin{theorem}[Lagrange's Theorem]
  Let $G$ be a finite group and $H \le G$, then $$|H| \mid |G|$$  
  and $$\frac{|G|}{|H|}$$ is the number of $H$-cosets in $G$.
\end{theorem}

\begin{definition}
  If $H \le G$. then the ``index of $H$ in $G$'' is the number of 
  left $H$ cosets in $G$
\end{definition}

\begin{corollary}
  If $G$ is a finite group and $x \in G$ then $|x| \mid |G|$, So, $x^|G|$ = 1
\end{corollary}

\begin{proof}
  let $H = \langle x \rangle \le G$ So $|x| \mid |G|$
  Since for $x^a = 1$ if and only if $|x| \mid a$, So $x^{|G|} = 1$
\end{proof}

\begin{corollary}
  If $|G| = p$ is prime, then $G \cong \ZZ/p\ZZ$
\end{corollary}

\begin{proof}
  Take any $x \in G \setminus \{1\}$, $|x| \mid |G|$, So, $|x| = p$
  Since $\langle x \rangle = H \le G$ and $p = |x| = |H|$ therefore $H = G$
\end{proof}

\begin{theorem}
  For any $n \in \NN$ either $p \mid n$ or $p \mid n^{p-1} -1$
\end{theorem}

\begin{theorem}[Sylow]
  If $G$ is finite of order $p^\alpha \cdot m$ where $p \nmid m$ where $p$ is prime.
  Then $G$ has a subgroup of size $p^\alpha$.
\end{theorem}

\begin{definition}
  If $H, K \le G$ then $$HK = \{hk \mid h \in H, k \in K\} = \bigcup_{h \in H} hK$$
\end{definition}

\begin{lemma}
  If $cK$ intersects $H$ then $|cK \cap H| = |K \cap H|$
\end{lemma}

\begin{proof}
  Let $a \in cK \cap H$ let $f : K \cap H \to cK \cap H$, $x \mapsto ax$

  Claim: $x\in K \cap H \implies ax \in cK \cap H$
  $ax \in H$ because $a, x \in H \le G$

  $a = cl$ for some $l \in K$ because $a \in cK$
  $ax = c\underbrace{(lx)}_{\in K} \in cK$
  So, $f$ is now injective

  Claim: If $y \in cK \cap H$ then $a^{-1}y \in K \cap H$
  $y \in cK$, $y = cl$, $a^{-1}y = (a^{-1}cl \in K$

\end{proof}

\begin{theorem}
  If $H, K$ are finite subgroups of $G$ then 
  \[|HK| = \frac{|H|\cdot|K|}{|H \cap K|}\]
\end{theorem}

\begin{proof}
  $|HK| = |\displaystyle\bigcup_{h \in H} hK| = |K| \cdot \text{number of $K$-coests of the form $hK$ for $h \in H$}$
  Each $h \in H$ define a coset $hK$
\end{proof}

\begin{proposition}
  $HK \le G$ if and only if $HK = KH$
\end{proposition}

\begin{proof}
  $\Leftarrow$
  \begin{itemize}
    \item $(hk)^{-1} = k^{-1}h^{-1} \in KH = HK$
    \item 
  \end{itemize}
  $\Rightarrow$, $H, K \le HK, H = H\cdot 1 \subseteq HK, K = 1\cdot K \subseteq HK$

  So, $KH \subseteq HK$

  Let $y \in HK, y = hk, y^{-1} = k^{-1}h^{-1} \in KH$

  So $HK \subseteq KH$
\end{proof}

\begin{theorem}[First Isomorphism Theorem]
  If $\varphi: G \to H$ is a homomorphism then
  \begin{itemize}
    \item $\ker(\varphi) \trianglelefteq G$
    \item $G/\ker(\varphi) \cong \varphi(G)$
  \end{itemize} 
\end{theorem}

\begin{definition}
  $\varphi(G) = \text{im}(\varphi) = \{y \in H \mid \exists x \in G, \varphi(x) = y\}$
\end{definition}
\begin{proof}
  Build $f: G \to G/\ker(\varphi)$, $a\cdot K \mapsto \varphi(a)$ ($K = \ker(\varphi)$)

  $aK = bK \iff b^{-1}a \in K$

  If $aK = bK$ want $\varphi(a) = \varphi(b)$

  \begin{align*}
    \varphi(a) &= \varphi(b \cdot b^{-1}a) \\
    &= \varphi(b) \cdot \varphi(b^{-1}a) \\
    &= \varphi(b) \cdot 1 \\
    &= \varphi(b)
  \end{align*}

  \begin{align*}
    f(aK \cdot bK) &= f(ab \cdot K) = 
  \end{align*}

\end{proof}

\begin{corollary}
  Let $\varphi: G \to H$ be a homomorphism
  \begin{enumerate}
    \item $\varphi$ is 1-to-1 iff $\ker(\varphi) = \{1_G\}$
    \item $|G/\ker(\varphi)| = |\varphi(G)|$
  \end{enumerate}
\end{corollary}

\begin{theorem}[2nd or ``Diamond'' isonorphism theorem]
  Given $G$, a group, $A, B \le G$ and $A \le N_G(B)$
  (i.e., $aBa^{-1} = B$ for every $a \in A$)
  then 
  \begin{itemize}
    \item $AB \le G$ 
    \item $B \unlhd AB$ 
    \item $A \cap B \unlhd A$
    \item $AB/B \cong A / (A \cap B)$
  \end{itemize}
\end{theorem}

\begin{proof}
  \text{ }
  \begin{itemize}
    \item 
  For $B \unlhd AB$.
  For any $a \in A, b \in B$
  \begin{align*}
    abB(ab)^{-1} &= B \\
    abB(ab)^{-1} &=a(bBb^{-1})a^{-1} = aBa^{-1} = B \\
  \end{align*}
  \item For $A \cap B \unlhd A$

  we want for any $a \in A$, $a(A \cap B) a^{-1} = A \cap B$
  \begin{align*}
    a(A \cap B) a^{-1} &\subseteq aAa^{-1} = A \\
    a(A \cap B) a^{-1} &\subseteq aBa^{-1} = B \\
  \end{align*}
  Given $y \in A \cap B$, for any $a \in A$, WANT $y \in a(A \cap B) a^{-1}$
  \begin{align*}
    a^{-1}ya &\in a^{-1}(A\cap B)A \subseteq A \cap B \\
    y = a(a^{-1}ya)a^{-1} &\in a(A \cap B)A^{-1}
  \end{align*}
  Therefore $a(A \cap B) a^{-1} = A \cap B$, so $A \cap B \unlhd A$
  \item WANT $\varphi: A \to AB/B$
  \begin{align*}
    x &\in ab \cdot B = aB \\
    x &= ab \cdot b' (\text{for some }b' \in B)\\
    &= a\cdot (bb')(\text{for some }b' \in B)
  \end{align*}
  $\varphi(a) = aB$
  \[A \to AB \to AB/B\]
  \[a \mapsto a \mapsto AB\]
  \begin{align*}
    \varphi(a \cdot a') &= a\cdot a' B \\
    \varphi(a)\cdot \varphi(a') &= aB\cdot a'B = aa'B
  \end{align*}
  $\varphi$ is onto. For any $ab \in AB$
  \end{itemize}

\end{proof}

\begin{theorem}[The 3rd theorem]
  Given $G$, a group, $H, K \unlhd G$ with $H \unlhd K$ Then
  $K/H \unlhd G/H$ and $(G/H)/(K/H) \cong(G/K)$
\end{theorem}